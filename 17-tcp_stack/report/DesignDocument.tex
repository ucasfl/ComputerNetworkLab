%        File: DesignDocument.tex
%     Created: 一 3月 26 01:00 下午 2018 C
% Last Change: 一 3月 26 01:00 下午 2018 C
%
\documentclass[UTF8,noindent]{ctexart}
\usepackage[a4paper,left=2.0cm,right=2.0cm,top=2.0cm,bottom=2.0cm]{geometry}
\usepackage{hyperref}
\usepackage{url}
\usepackage{graphicx}
\usepackage{amsmath}
\usepackage{amssymb}
\usepackage{enumitem}
\usepackage{tikz}
\usepackage{float}
\usepackage{xeCJK}
\usepackage{listings}
\usepackage{xcolor}
\lstset{language = c,numbers=left, showstringspaces=false,keywordstyle= \color{ blue!70 },commentstyle=\color{red!50!green!50!blue!50}, frame=shadowbox, rulesepcolor= \color{ red!20!green!20!blue!20 } 
} 
\CTEXsetup[format={\Large\bfseries}]{section}
\usetikzlibrary{graphs}
%\newtheorem*{lemma}{Lemma}
\title{\CJKfamily{zhkai}计算机网络研讨课实验报告}
\author{{\CJKfamily{zhkai}冯吕}\ $2015K8009929049$}
\date{\today}
\begin{document}
\maketitle
\zihao{5}
\CJKfamily{zhsong}
%\begin{center}
%  \begin{tabular}{|p{15cm}|}
%    \hline
\section{{\CJKfamily{zhhei}实验题目}}网络传输机制实验三
%\hline
\section{{\CJKfamily{zhhei}实验内容}}
在本次实验中,需要通过超时重传机制实现有丢包环境下的可靠传输;
\section{{\CJKfamily{zhhei}实验流程}}
\subsection{增加$tcp\_sock$结构}
在有丢包环境下,当一个包发送出去之后,不能直接丢弃,而需要先保存下来,直到$ACK$之后才能删除,同时,需要进行超时重传,因此,在$tcp\_sock$中增加如下三个结构:
\begin{lstlisting}
struct tcp_sock{
  ...
  struct tcp_timer retrans_timer;
  struct list_head send_buf;
  struct list_head ofo_buf;
  ...
};
\end{lstlisting}
所有发送出去未确认的包需要先保存在$send\_buf$中,同时,当收到不连续的包时,先保存到$ofo\_buf$中,当发送一个包出去之后,启动定时器。

存储于$send\_buf$和$ofo\_buf$中的数据结构如下:
\begin{lstlisting}
struct send_packet{
	struct list_head list;
	char *packet;
	int len;
};

struct ofo_packet{
	struct list_head list;
	char *packet;
	int len;
	int seq_num;
};
\end{lstlisting}

\subsection{修改$tcp\_send\_packet$}
在发送包的时候,需要将包保存到$send\_buf$中,因此,需要修改$tcp\_send\_packet$函数,将包保存下来,同时启动定时器:
\begin{lstlisting}
void tcp_send_packet(struct tcp_sock *tsk, char *packet, int len) 
{
    ...

	struct send_packet *send_pkt = (struct send_packet 
	*)malloc(sizeof(struct send_packet));
	send_pkt->packet = (char *)malloc(len);
	send_pkt->len = len;
	memcpy(send_pkt->packet, packet, len);
	list_add_tail(&send_pkt->list, &tsk->send_buf);
	tcp_set_retrans_timer(tsk);

	ip_send_packet(packet, len);
}
\end{lstlisting}

相应地,$tcp\_send\_control\_packet$函数也需要修改:如果发送的包为$TCP\_SYN\mid TCP\_ACK$包,则将其保存到$send\_buf$中,启动定时器;

\subsection{删除$ACK$数据}
当收到$ACK$数据包之后,将$send\_buf$中$seq\_end <ack$的数据包删除,同时关闭定时器:
\begin{lstlisting}
void remove_ack_data(struct tcp_sock *tsk, int ack_num){
	tcp_remove_retrans_timer(tsk);
	struct send_packet *pos, *q;
	list_for_each_entry_safe(pos, q, &tsk->send_buf, list){
		struct tcphdr *tcp = packet_to_tcp_hdr(pos->packet);
		struct iphdr *ip = packet_to_ip_hdr(pos->packet);
		if (ack_num >= ntohl(tcp->seq)){
			tsk->snd_wnd += (ntohs(ip->tot_len) - IP_HDR_SIZE(ip) 
			- TCP_HDR_SIZE(tcp));
			free(pos->packet);
			list_delete_entry(&pos->list);
		}
	}
	if (!list_empty(&tsk->send_buf)){
		tcp_set_retrans_timer(tsk);
	}
}
\end{lstlisting}

\subsection{重传实现}
本次实验中,需要通过定时器重传,超时之后,重新发送,如果重传次数超过三次依旧失败,则关闭连接。

设置定时器与关闭定时器:
\begin{lstlisting}
void tcp_set_retrans_timer(struct tcp_sock *tsk){
	struct tcp_timer *timer = &tsk->retrans_timer;

	timer->type = 1;
	timer->timeout = TCP_RETRANS_INTERVAL;
	timer->retrans_number = 0;

	list_add_tail(&timer->list, &timer_list);
}

void tcp_remove_retrans_timer(struct tcp_sock *tsk){
	list_delete_entry(&tsk->retrans_timer.list);
}
\end{lstlisting}

在$tcp\_timer$结构中增加一项记录重传次数,发送数据包时启动定时器。需要修改$tcp\_scan\_timer\_list$:
\begin{lstlisting}
void tcp_scan_timer_list()
{
	struct tcp_sock *tsk;
	struct tcp_timer *t, *q;
	list_for_each_entry_safe(t, q, &timer_list, list) {
		t->timeout -= TCP_TIMER_SCAN_INTERVAL;
		if (t->timeout <= 0 && t->type == 0) {
			list_delete_entry(&t->list);

			// only support time wait now
			tsk = timewait_to_tcp_sock(t);
			if (! tsk->parent)
				tcp_bind_unhash(tsk);
			tcp_set_state(tsk, TCP_CLOSED);
			free_tcp_sock(tsk);
		}
		else if(t->timeout <= 0 && (t->type == 1)){
			tsk = retranstimer_to_tcp_sock(t);
            if(t->retrans_number == 3){
            	tcp_sock_close(tsk);
                //free_tcp_sock(tsk);
                return ;
            }
            struct send_packet *buf_pac = list_entry(tsk->send_buf.next, 
			struct send_packet, list);

            if(!list_empty(&tsk->send_buf)){
                char *packet = (char *)malloc(buf_pac->len);
                memcpy(packet, buf_pac->packet, buf_pac->len);
                /*struct tcphdr *tcp = packet_to_tcp_hdr(packet);*/
                /*struct iphdr *ip = packet_to_ip_hdr(packet);*/

                ip_send_packet(packet, buf_pac->len);
                t->timeout = TCP_RETRANS_INTERVAL;

			    for(int i = 0; i <= t->retrans_number; i++){
			    	t->timeout *= 2;
			    }
			    t->retrans_number ++;
			}
		}
	}
}
\end{lstlisting}
  \begin{itemize}
	\item (1)判断定时器类型,如果类型为$wait$,则根据是否$timeout$关闭即可,如果类型为$retrans$,转(2);
	\item (2)判断重传次数是否小于$3$,如果不小于,则关闭该$timer$对应的$socket$,否则转(3);
	\item (3)重传$send\_buf$中的第一个包,更新定时器:$timeout*2, retrans\_number + 1$
  \end{itemize}


\section{{\CJKfamily{zhhei}实验结果}}
能够建立连接并传输数据,但有时候会传输失败。

\section{{\CJKfamily{zhhei}结果分析}}
可能会出现连续三次都丢包的情况,导致数据传输失败。

\end{document}


